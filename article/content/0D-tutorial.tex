\section{Tutorial}

\subsection{How to draw a tree}
\begin{figure}[H]
    \centering
    \begin{tikzpicture}[level distance=1.5cm,
      level 1/.style={sibling distance=3cm},
      level 2/.style={sibling distance=1.5cm}]
      \node [treenode, draw] {$1.0$}
        child {node [treenode, draw] {$0.5$}
            child {node [treenode, draw]{$0.1$}}
            child {node [treenode, draw]{$0.4$}}
        }
        child { node [treenode, draw] {$0.5$}
            child {node [treenode, draw]{$0.2$}}
            child {node [treenode, draw]{$0.3$}}
        };
    \end{tikzpicture}
    \caption{I am a Tree}
    \label{fig:my_label}
\end{figure}


\subsection{How to write pseudocode}

\begin{algorithm}[H]
	\caption{Find all bridge edges in $G$} 
	\begin{algorithmic}[1]
    \State $cnt = 0$
    \State initialize empty arrays $pre[V], low[V],vis[V]$
	 \Procedure{dfs}{u}
     		\State $vis[u] = true$
            \State $pre[u]=low[u]=cnt++$
            \For{every edge $(u \rightarrow v) \in E$ }
            		\If{$vis[v]==true$}
                    	\State $low[u]=\min(low[u],pre[v])$
                      \Else
                      	\State $dfs(v)$
                      	\State $low[u] =\min(low[u], low[v]) $
                        \If{$low[v] > pre[u]$}
                        	\State $u\rightarrow v$ is a bridge edge
                        \EndIf
                    \EndIf
            \EndFor
     \EndProcedure
	\end{algorithmic} 
	
\end{algorithm}

\subsection{How to insert figures}

\begin{figure}[H] % 不用[H]的话图片可能会跑到其它地方去
    \centering
    \includegraphics[width=0.8\textwidth]{figure/asuka2.png}
    \caption{Asuka}
    \label{fig:Asuka}
\end{figure}


\subsection{Equations, theorems}

\begin{equation} \label{eq1}
    I(X;Y) = H(X) - H(X|Y)
\end{equation}


\begin{theorem}[Theorem 1]
Let \(f\) be a function whose derivative exists in every point, then \(f\) is 
a continuous function.
\end{theorem}

\begin{theorem}[Pythagorean theorem]
\label{pythagorean}
This is a theorem about right triangles and can be summarised in the next 
equation 
\[ x^2 + y^2 = z^2 \]
\end{theorem}

And a consequence of theorem \ref{pythagorean} is the statement in the next 
corollary.

\begin{corollary}
There's no right rectangle whose sides measure 3cm, 4cm, and 6cm.
\end{corollary}

You can reference theorems such as \ref{pythagorean} when a label is assigned.

\begin{lemma}
Given two line segments whose lengths are \(a\) and \(b\) respectively there is a 
real number \(r\) such that \(b=ra\).
\end{lemma}

\subsection{citing}
To cite, simply \cite{homo114514}




If you want to add mathematical equations, this may be done either this way: $a +  b \neq \frac{a}{b}$. You may also add Greek letters like this: $\alpha$.\\ 

\blindtext % Some blind text

% Including figures
\begin{figure}[htpb!] % Defines figure environment
    \centering % Centers your figure
\includegraphics[scale=0.8]{figure/figure.png} % Includes your figure and defines the size
    \caption{A circle} % For your caption
    \label{fig:my_label} % If you want to label your figure for in-text references
\end{figure}

\blindtext % Some blind text

% Including tables
%   Simple table
\begin{table}[] % Add htpb! to make sure that table is where it should be
    \centering
    \begin{tabular}{c|c}
        Saturday & Sunday \\
        12 & 18
    \end{tabular}
    \caption{Overview of the weekend}% Caption for tables
    \label{tab:weekend} % Reference for in-line referencing
\end{table}

%   Table with table generator
\begin{table}[] % Add htpb! to make sure that table is where it should be
    \centering
\begin{tabular}{@{}lll@{}}
\toprule
  & A   & B \\ \midrule
C & 100 & 2 \\
D & 3   & 5 \\ \bottomrule
\end{tabular}
    \caption{Random numbers} % Caption for tables
    \label{tab:numbers} % Reference for in-line referencing
\end{table}


\begin{figure}[H]
    \centering
    \begin{tikzpicture}[level distance=1.5cm,
      level 1/.style={sibling distance=3cm},
      level 2/.style={sibling distance=1.5cm}]
      \node [treenode, draw] {$1$}
        child {node [treenode, draw] {$5$}
        }
        child { node [treenode, draw] {$3$}
            child {node [treenode, draw]{$2$}}
            child { node [treenode, draw]{$4$} edge from parent [dashed]}
        };
    \end{tikzpicture}
    %\caption{I am a Tree}
    \label{fig:gbscProof2}
\end{figure}