%----------------------------------------------------------------------------------------
%	Packages
%----------------------------------------------------------------------------------------

% Necessary
\usepackage[german,english]{babel} % English and German language 
\usepackage{booktabs} % Horizontal rules in tables 
% For generating tables, use “LaTeX” online generator (https://www.tablesgenerator.com)
\usepackage{comment} % Necessary to comment several paragraphs at once
\usepackage[utf8]{inputenc} % Required for international characters
\usepackage[T1]{fontenc} % Required for output font encoding for international characters

% Might be helpful
\usepackage{amsmath,amsfonts,amsthm} % Math packages which might be useful for equations
\usepackage{tikz} % For tikz figures (to draw arrow diagrams, see a guide how to use them)
\usepackage{tikz-cd}
\usetikzlibrary{positioning,arrows} % Adding libraries for arrows
\usetikzlibrary{decorations.pathreplacing} % Adding libraries for decorations and paths
\usepackage{tikzsymbols} % For amazing symbols ;) https://mirror.hmc.edu/ctan/graphics/pgf/contrib/tikzsymbols/tikzsymbols.pdf 
\usepackage{blindtext} % To add some blind text in your paper


%---------------------------------------------------------------------------------
% Additional settings
%---------------------------------------------------------------------------------

%---------------------------------------------------------------------------------
% Define your margins
\usepackage{geometry} % Necessary package for defining margins

\geometry{
	top=2cm, % Defines top margin
	bottom=2cm, % Defines bottom margin
	left=2cm, % Defines left margin
	right=2cm, % Defines right margin
	%includehead, % Includes space for a header
	%includefoot, % Includes space for a footer
	%showframe, % Uncomment if you want to show how it looks on the page 
}

\setlength{\parindent}{15pt} % Adjust to set you indent globally 

%---------------------------------------------------------------------------------
% Define your spacing
\usepackage{setspace} % Required for spacing
% Two options:
\linespread{1.5}
%\onehalfspacing % one-half-spacing linespread

%----------------------------------------------------------------------------------------
% Define your fonts
\usepackage[T1]{fontenc} % Output font encoding for international characters
\usepackage[utf8]{inputenc} % Required for inputting international characters

%\usepackage{palatino} % Use the XCharter font


%---------------------------------------------------------------------------------
% Define your headers and footers

%\usepackage{fancyhdr} % Package is needed to define header and footer
%\pagestyle{fancy} % Allows you to customize the headers and footers

%\renewcommand{\sectionmark}[1]{\markboth{#1}{}} % Removes the section number from the header when \leftmark is used

% Headers
%\lhead{} % Define left header
%\chead{\textit{}} % Define center header - e.g. add your paper title
%\rhead{} % Define right header

% Footers
%\lfoot{} % Define left footer
%\cfoot{\footnotesize \thepage} % Define center footer
%\rfoot{ } % Define right footer

%---------------------------------------------------------------------------------
%	Add information on bibliography
%\usepackage{natbib} % Use natbib for citing
%\usepackage{har2nat} % Allows to use harvard package with natbib

% https://mirror.reismil.ch/CTAN/macros/latex/contrib/har2nat/har2nat.pdf

% For citing with natbib, you may want to use this reference sheet: 
% http://merkel.texture.rocks/Latex/natbib.php


% 在下面引用需要使用的包

\usepackage{tikz}
\usetikzlibrary{trees}
%\usepackage{ctex}
\usepackage{amsmath,amscd,amsbsy,amssymb,latexsym,url,bm,amsthm}
\usepackage{epsfig,graphicx}
\usepackage{enumitem,balance}
\usepackage{wrapfig}
\usepackage{mathrsfs,euscript}
\usepackage{hyperref}

\usepackage{float}
\usepackage{geometry}
\usepackage{listings}
\usepackage{physics}

\usepackage{amssymb}

\usepackage{algorithm} 
\usepackage{algpseudocode}
\usepackage{amsmath,amscd,amsbsy,amssymb,latexsym,url,bm}
\usepackage{epsfig,graphicx}
\usepackage{enumitem,balance}
\usepackage{wrapfig}
\usepackage{mathrsfs,euscript}
\usepackage{hyperref}

\usepackage{float}
\usepackage{geometry}
\usepackage{listings}
\usepackage{physics}


\usepackage{algorithm} 
\usepackage{algpseudocode} 
\usepackage[group-separator={,}]{siunitx}
\usepackage{subcaption}

% 画图设置

\tikzset{
treenode/.style = {circle, minimum size=#1,inner sep=2pt, outer sep=0pt},
%treenode/.style = {}     
treenode/.default = 25pt % size of the circle diameter 
}

\tikzset{
rectnode/.style = {rectangle,minimum size=#1,inner sep=4pt, outer sep=0pt},
%treenode/.style = {}     
rectnode/.default = 20pt 
}


\usetikzlibrary{matrix}
\tikzset{r/.style={fill=red}}
\tikzset{b/.style={fill=cyan}}
\tikzset{w/.style={fill=white}}

\tikzset{boardstyle/.style={matrix of nodes,
        nodes in empty cells,
        row sep=-\pgflinewidth,
        column sep=-\pgflinewidth,
        nodes={draw,minimum width=0.3cm,minimum height=0.3cm,anchor=center}}}

% 宏定义

\providecommand{\keywords}[1]{\textbf{\textit{Keywords:}} #1}

\newtheorem{theorem}{Theorem}
\newtheorem{corollary}{Corollary}
\newtheorem{lemma}{Lemma}

% 这是我自己添加的
\newtheorem{example}{Example}
\newtheorem{proposition}{Proposition}
\newtheorem{definition}{Definition}


% fonts
\usepackage[sc]{mathpazo}